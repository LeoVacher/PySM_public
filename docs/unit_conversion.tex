\title{Unit conversion in PySM}
\author{Ben Thorne}
\date{\today}

\documentclass[12pt]{article}
\usepackage{amsmath}
\begin{document}
\maketitle

\section{Introduction}

PySM uses input templates specified in $\mu K_{\rm RJ}$, and scales them to other
frequencies using scaling laws appropriate for $\mu K_{\rm RJ}$. This is fine in
the case of delta-bandpasses, as frequency dependent unit conversions can be
applied after scaling to produce the same result, no matter which unit the
scaling was done in. However, for bandpass-integrated quantities, the matter of
unit conversion requires a little clarification. Here we describe the procedure
used in PySM.

\section{Unit definitions}

\subsection{$K_{\rm CMB}$}

References: Dodelson page 380, Planck HFI Spectral response paper. 

The spectrum of the CMB brightness is a black body, $B(\nu, T)$. This has been well measured,
and is usually removed from observations of CMB anisotropies. The brightness of anisotropies
have a different spectrum; they are small perturbations to the temperature of
the CMB and so we can Taylor expand the black body function:
\[
B_{\rm anisotropy}(\nu, T + \Delta T) = B(\nu, T_{\rm CMB}) + \frac{\partial B(\nu, T)}{\partial T}
\bigg\rvert_{T_{\rm CMB}\Delta T}
\]
As mentioned, we usually are not interested in the spectrum of the monopole,
so we remove it:
\[
B_{\rm anisotropy}(\nu, T+\Delta T) = \frac{\partial B(\nu, T)}{\partial T}\bigg\rvert_{T_{\rm CMB}
  \Delta T}.
\]
From now on we will write: $b^\prime_\nu \equiv \frac{\partial B(\nu, T)}{\partial T}\bigg\rvert_{T_{\rm CMB}}$.
Given some observed signal, $I_\nu$, we then define the temperature in CMB units, $T_{\rm CMB}$, to be:
\[
T_{\rm CMB} b^\prime_\nu = I_\nu.   
\]
So an observation of the true CMB anisotropy spectrum is flat in CMB units.

\subsection{$K_{\rm RJ}$}

Rayleigh-Jeans units are often useful in radio astronomy as they are more human-readable than
flux units. In the Rayleigh-Jeans limit the black body function is:
\[
B_{\rm RJ}(\nu, T) = \frac{2 k_B \nu^2 T}{c^2}
\]
We use this to define the Rayleight-Jeans temperature, $T_{\rm RJ}$, for a given observation of spectral
radiance $I_\nu$ as:
\[
\frac{2 k_B \nu T_{\rm RJ}}{c^2} = I_\nu.
\]

\subsection{${\rm Jy / sr}$}
The incident power on the detector in some band defined by the bandpass $\tau(\nu)$ is:
\[
P = \int_0^\infty d\nu \ \tau(\nu) I_\nu (A \Omega)_\nu.
\]

It is clear from this that the incident signal $I_\nu$ in SI units must have units of:
$[{\rm W m^{-2}Hz^{-1} sr^{-1}}]$, in order for the RHS to give units of Watts. This is a flux density
per unit solid angle.  We define the Jansky unit:
\[
1 {\rm Jy} = 10^{-26} \frac{{\rm W}}{{\rm m^2 Hz}},
\]
which is a unit of flux density, obtained by integrating over the source solid angle. This is
therefore most appropriate for point sources. However, the Jansky can also be used to described
surface brightness of extended sources by dividing by the source solid angle we consider (e.g.
the beam of the telescope), hence the unit ${\rm Jy / sr}$.

\section{Delta bandpasses}

Converting between these units depends on frequency. In the case of a delta bandpass the frequency
is obvious, and so we can evaluate the conversion factor simply. For example to convert a map
in $\mu K_{\rm RJ}$ to $\mu K_{\rm CMB}$ we write:
\[
U_{\rm RJ \rightarrow CMB} T_{\rm RJ} = T_{\rm CMB} 
\]
so:
\[
\begin{aligned}
  U_{\rm RJ \rightarrow CMB} &= \frac{I_\nu / b^\prime_\nu}{I_\nu / (\frac{2 k_B}{c^2})} \\
  &= \frac{2 k_B / c^2}{b^\prime_\nu}
\end{aligned}
\]

\section{Arbitrary bandpasses}

When we integrate over a bandpass there is no longer an obvious frequency at which we should
evaluate the freuqency dependent conversion factors. Furthermore, since the physical quantity
we observe is the power hitting the detector, what does the CMB and RJ units even mean?

To answer this let's consider the power hitting the detector, $i$, making a map, $m_i$:
\[
m_i = G_i \int^\infty_0 d\nu \ \tau(\nu) (A \Omega) I_\nu.
\]
Where $G$ is a detector-specific factor accouting for various effects, $\Omega$ is the solid
angle subtended by the beam, and $A$ is the area of the detector. In this equation $I_\nu$ is
in SI units, as we will have the power in Watts. We can amalgamate the factors that do not
depend on frequency, and drop the $i$ subscript, since the weighted average
of detector maps can be expressed in an equivalent form:
\[
m = A \int^\infty_0 d\nu \ \tau(\nu) I_\nu.
\]
In order to determine the value of the constant $A$ we need to calibrate, this is often
done using the CMB dipole, meaning our map $m$ will be in units of CMB temperature. Assume
that the observed source has a CMB dipole spectrum, then if we were to observe the true
CMB our map should just be of the CMB temperature:
\[
T_{\rm CMB} =  A \int^\infty_0 d\nu \ \tau(\nu) T_{\rm CMB} b^\prime_\nu.
\]
So:
\[
A = 1 / \int^\infty_0 d\nu \ \tau(\nu) b^\prime_\nu.
\]
Therefore a map of the sky in CMB units is given by:
\[
m = \int^\infty_0 d\nu \ \tau(\nu) I_\nu / \int^\infty_0  d\nu \ \tau(\nu) b^\prime_\nu .
\]
At this point $I_\nu$ is in SI units, but since we will be simulating in ${\rm Jy/sr}$ we want
to change this:
\[
m_{\rm CMB} =\frac{ \int^\infty_0 d\nu \ \tau(\nu) I^{\rm Jy/sr}_\nu d\nu}{ \int^\infty_0 \ \tau(\nu) 10^{-20} b^\prime_\nu}.
\]
By similar arguments:
\[
\begin{aligned}
  m_{\rm RJ} &=\frac{ \int^\infty_0 d\nu \ \tau(\nu) I^{\rm Jy/sr}_\nu d\nu}{ \int^\infty_0 \ \tau(\nu) 10^{-20} \frac{2 k_B \nu^2}{c^2}} \\
  m_{\rm Jy/sr} &= \int^\infty_0 d\nu \ \tau(\nu) I^{\rm Jy/sr}_\nu d\nu
\end{aligned}
\]
For a given bandpass and output unit, PySM computes the appropriate integrals over unit conversions,
computes the signal in ${\rm Jy/sr}$ over the bandpass and integrates this, and returns maps as defined
here.
\end{document}
